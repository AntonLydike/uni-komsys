\documentclass{article}
\usepackage{microtype}
\usepackage[utf8]{inputenc} 
\usepackage[a4paper, total={6in, 9.6in}]{geometry}
\usepackage{MnSymbol}
\usepackage{enumerate}
\usepackage{amsmath}
\usepackage{fancyhdr}
\usepackage{xcolor}
\usepackage{tikz}
\usepackage{pgfplots}
\usepackage{marvosym}
\usepackage{svg}

\widowpenalties=4 10000 10000 150 0

%% headers
\pagestyle{fancy}
\fancyhf{}
\rhead{Kommunikationssysteme WS19/20}
\lhead{Daniel Schubert, Anton Lydike}
\rfoot{Seite \thepage}

% simple command to display Aufgabe <num>)       ___ / <num>p.
\newcommand\task[1]{\section*{Aufgabe #1)\hfill \underline{\,\,\,\,\,\,}\,\,/1p.}}

% Interpretation (I)
\newcommand\I{I}
% Interpretation und belegung (I, \beta)
\newcommand\Ib{\I, \beta}

%% models
\newcommand\lmodels{\leftmodels} 			% =|
\newcommand\bimodels{\leftmodels\models}	% =||=


%% table for total points
\newcommand\pointsttl[1]{\section*{Gesamtpunkte: \hfill \underline{\,\,\,\,\,\,}\,\,/#1p.}}

%% Funktionen und Prädikate
% Funktionen (arg ist anzahl der stellen)
\newcommand\func[1]{\mathcal{F}^{#1}}
% Prädikate (arg ist anzahl der stellen)
\newcommand\praed[1]{\mathcal{P}^{#1}}

%% Regeln
\newcommand\defrule[2]{\frac{#1}{#2}}

%% Funktionszahl
\newcommand\funcnum[1]{\#_{F}\, #1}

% Für ersetzungen in belegungen wie { x \mapsto d }
\newcommand\repl[2]{\{#1 \mapsto #2\}}

% für alle x .
\newcommand\fall[1]{\forall #1 \, . \,}
\newcommand\ex[1]{\exists #1 \, . \,}

% short biimplication
\newcommand\biimpl{\Leftrightarrow}

% draw a box on the right side of the page
\newcommand\qed{ \hfill $\Box$ }

% red, green, blue text:

\definecolor{greeen}{RGB}{34,139,34}

\newcommand\red[1]{\textcolor{red}{#1}}
\newcommand\green[1]{\textcolor{greeen}{#1}}
\newcommand\blue[1]{\textcolor{blue}{#1}}

% more symbols: https://oeis.org/wiki/List_of_LaTeX_mathematical_symbols

\newcommand\cfgtitle[1]{\title{\vspace{-1.5cm}Übungsblatt #1\\%
\begin{large} Übungsgruppe Metcalfe \end{large}} \lfoot{Übungsblatt #1}\cfoot{Übungsgruppe Metcalfe}}
\author{Daniel Schubert\\Anton Lydike}


%%%%%%%%%%%%%%%%%%%%%%%
%% plotting helpers  %%
%%%%%%%%%%%%%%%%%%%%%%%

%% these draw vertical features
\newcommand\htl[1]{(#1,1) (#1,-1)}  		%% draw line from low to high
\newcommand\lth[1]{(#1,-1) (#1,1)}			%% draw line from high to low

\newcommand\sigtick[2]{\htl{#1} \lth{#2}}	%% draw a htl and then lth line

%% these draw horizontal features
\newcommand\sig[3]{(#2,#1) (#3,#1)}		%% draw a line at height #1 from x=#2 to x=#3
\newcommand\sighi[2]{\sig{1}{#1}{#2}}		%% draw a high signal from #1 to #2
\newcommand\sigmed[2]{\sig{0}{#1}{#2}}		%% draw a null signal from #1 to #2
\newcommand\siglo[2]{\sig{-1}{#1}{#2}}		%% draw a low  signal from #1 to #2


\newcommand\fakeaxis[2]{\addplot [-stealth,black] coordinates {(#1,0) (#2,0)};}



%% units
\newcommand\m{\text{ m}}
\newcommand\s{\text{ s}}
\newcommand\mps{\frac{\text{m}}{\text{s}}}
\newcommand\Gbps{\text{ Gbps}}
\newcommand\bps{\text{ bps}}
\newcommand\bit{\text{ b}}
\newcommand\B{\text{ B}}
