\documentclass{article}
\usepackage{microtype}
\usepackage[utf8]{inputenc} 
\usepackage[a4paper, total={6in, 9.6in}]{geometry}
\usepackage{MnSymbol}
\usepackage{enumerate}
\usepackage{amsmath}
\usepackage{fancyhdr}
\usepackage{xcolor}
\usepackage{tikz}
\usepackage{pgfplots}
\usepackage{marvosym}

\widowpenalties=4 10000 10000 150 0

%% headers
\pagestyle{fancy}
\fancyhf{}
\rhead{Kommunikationssysteme WS19/20}
\lhead{Daniel Schubert, Anton Lydike}
\rfoot{Seite \thepage}

% simple command to display Aufgabe <num>)       ___ / <num>p.
\newcommand\task[1]{\section*{Aufgabe #1)\hfill \underline{\,\,\,\,\,\,}\,\,/1p.}}

% Interpretation (I)
\newcommand\I{I}
% Interpretation und belegung (I, \beta)
\newcommand\Ib{\I, \beta}

%% models
\newcommand\lmodels{\leftmodels} 			% =|
\newcommand\bimodels{\leftmodels\models}	% =||=


%% table for total points
\newcommand\pointsttl[1]{\section*{Gesamtpunkte: \hfill \underline{\,\,\,\,\,\,}\,\,/#1p.}}

%% Funktionen und Prädikate
% Funktionen (arg ist anzahl der stellen)
\newcommand\func[1]{\mathcal{F}^{#1}}
% Prädikate (arg ist anzahl der stellen)
\newcommand\praed[1]{\mathcal{P}^{#1}}

%% Regeln
\newcommand\defrule[2]{\frac{#1}{#2}}

%% Funktionszahl
\newcommand\funcnum[1]{\#_{F}\, #1}

% Für ersetzungen in belegungen wie { x \mapsto d }
\newcommand\repl[2]{\{#1 \mapsto #2\}}

% für alle x .
\newcommand\fall[1]{\forall #1 \, . \,}
\newcommand\ex[1]{\exists #1 \, . \,}

% short biimplication
\newcommand\biimpl{\Leftrightarrow}

% draw a box on the right side of the page
\newcommand\qed{ \hfill $\Box$ }

% red, green, blue text:

\definecolor{greeen}{RGB}{34,139,34}

\newcommand\red[1]{\textcolor{red}{#1}}
\newcommand\green[1]{\textcolor{greeen}{#1}}
\newcommand\blue[1]{\textcolor{blue}{#1}}

% more symbols: https://oeis.org/wiki/List_of_LaTeX_mathematical_symbols

\newcommand\cfgtitle[1]{\title{\vspace{-1.5cm}Übungsblatt #1\\%
\begin{large} Übungsgruppe Metcalfe \end{large}} \lfoot{Übungsblatt #1}\cfoot{Übungsgruppe Metcalfe}}
\author{Daniel Schubert\\Anton Lydike}


%%%%%%%%%%%%%%%%%%%%%%%
%% plotting helpers  %%
%%%%%%%%%%%%%%%%%%%%%%%

%% these draw vertical features
\newcommand\htl[1]{(#1,1) (#1,-1)}  		%% draw line from low to high
\newcommand\lth[1]{(#1,-1) (#1,1)}			%% draw line from high to low

\newcommand\sigtick[2]{\htl{#1} \lth{#2}}	%% draw a htl and then lth line

%% these draw horizontal features
\newcommand\sig[3]{(#2,#1) (#3,#1)}		%% draw a line at height #1 from x=#2 to x=#3
\newcommand\sighi[2]{\sig{1}{#1}{#2}}		%% draw a high signal from #1 to #2
\newcommand\sigmed[2]{\sig{0}{#1}{#2}}		%% draw a null signal from #1 to #2
\newcommand\siglo[2]{\sig{-1}{#1}{#2}}		%% draw a low  signal from #1 to #2


\newcommand\fakeaxis[2]{\addplot [-stealth,black] coordinates {(#1,0) (#2,0)};}



%% units
\newcommand\m{\text{ m}}
\newcommand\s{\text{ s}}
\newcommand\mps{\frac{\text{m}}{\text{s}}}
\newcommand\Gbps{\text{ Gbps}}
\newcommand\bps{\text{ bps}}
\newcommand\bit{\text{ b}}
\newcommand\B{\text{ B}}


\usepackage{multicol,tabularx}
\usepackage{graphicx}
\usepackage{float}
\renewcommand{\arraystretch}{1.5}

\cfgtitle{6}
\date{Mittwoch 4.12.2019}


\begin{document}
\maketitle
\thispagestyle{fancy}

\task{1}

\begin{enumerate}[a)]
	\item \hfill 
		\begin{center}
			\includesvg[width=0.8\textwidth]{./the_internet.svg}
		\end{center}

	\item ,,So gut wie möglich'' das paket mit den aktuell vorhandenen Ressourcen zustellen.
	\item \begin{itemize}
		\item Wahr
		\item Falsch
		\item Wahr
		\item Wahr
	\end{itemize}
	\item \begin{itemize}
		\item \textit{Time To Live (TTL)} bezeichnet die Anzahl der "Hops" durch Knoten, bevor das Package nicht mehr weitergeleitet wird. Das TTL-Feld wird von jedem Router dekrementiert.
		\item Das \textit{Protocol}-Feld enthält die Nummer des Protokolls der höheren Schicht, zu dem die im IPv4-Paket transportierten Nutzdaten gehören.
	\end{itemize}
	\item \begin{itemize}
		\item Das \textit{Checksum}-Feld fällt weg. Dieses musste bei jedem Hop neu berechnet werden, da das TTL-Feld dekrementiert wird.
		\item Starke vereinfachung des Paketaufbaus, nur 8 statt 13 header Feldern.
	\end{itemize}
\end{enumerate}


\task{2}

\begin{enumerate}[a)]
	\item \begin{itemize}
		\item Verbindung Host A – R1:
		
		\begin{tabular}{c|c|c|c|c}
		\textit{Paket / Fragment} & \textit{Total Length} & \textit{Identification} & \textit{MF Flag} & \textit{Fragment Offset} \\ \hline
		1 & 1480 Byte & $x$ & 0 & 0 
		\end{tabular}	
		
		
		\item Verbindung Host R1 – R2:
		
		\begin{tabular}{c|c|c|c|c}
		\textit{Paket / Fragment} & \textit{Total Length} & \textit{Identification} & \textit{MF Flag} & \textit{Fragment Offset} \\ \hline
		1.1 & 1220 Byte & $x$ & 1 & 0 \\
		1.2 &  300 Byte & $x$ & 0 & 150 
		\end{tabular}
		
		\item Verbindung Host R2 – B:
		
		\begin{tabular}{c|c|c|c|c}
		\textit{Paket / Fragment} & \textit{Total Length} & \textit{Identification} & \textit{MF Flag} & \textit{Fragment Offset} \\ \hline
		1.1.1 & 580 Byte & $x$ & 1 & 0 \\
		1.1.2 & 580 Byte & $x$ & 1 & 70 \\
		1.1.3 & 120 Byte & $x$ & 0 & 140 \\
		1.2   & 300 Byte & $x$ & 0 & 150 
		
		\end{tabular}	
	
	\end{itemize}
	
	\item Wenn die \textit{MF-Flag} gesetzt ist, oder das \textit{Fragment-Offset} nicht null ist.
	\item Fragmentierung ist während der Übertragung aus Performance-Gründen nicht mehr erlaubt. Nur der Absender darf Pakete fragmentieren: Hierfür wird der ,,Fragment'' Extension Header verwendet
\end{enumerate}


\task{3}

\begin{enumerate}[a)]
	\item Sende \textit{ICMP Echo-Request-Packages} mit \textit{TTL} $= 1,\dots,n$ wobei $n$ entweder die Anzahl der hops bis zum Zielhost, oder die maximale Anzahl der hops ist, je nach dem, welcher wert kleiner ist. Anhand der \textit{Echo-Packages}, die entstehen, wenn das jeweilige Package das ende seiner \textit{TTL} erreicht, kann die verwendete Route nachvollzogen werden.

	\item \begin{itemize}
		\item Wahr
		\item Falsch
		\item Falsch
		\item Wahr
	\end{itemize}
	
	\item \begin{itemize}
		\item \begin{itemize}
			\item \textit{Echo-Request} (Type 8, Code 0)
			\item \textit{Echo-Reply} (Type 0, Code 0)
			\item \textit{Echo-Request} (Type 8, Code 0)
			\item \textit{Echo-Reply} (Type 0, Code 0)
		\end{itemize}
		\item \texttt{traceroute -I -w 10 -q 1 -m 5 192.168.5.1}
		
		\begin{tabular}{|r|l|r|l|}
			\hline
			\multicolumn{2}{|c|}{\textbf{1. Nachricht}} & \multicolumn{2}{c|}{\textbf{Antwort}} \\ \hline
			Sendeadresse:    & 192.168.1.1 & Sendeadresse:    & 192.168.1.254 \\ \hline
			Empfangsadresse: & 192.168.5.1 & Empfangsadresse: & 192.168.1.1 \\ \hline
			Nachrichtentyp:  & ICMP        & Nachrichtentyp:  & ICMP \\ \hline
			TTL-Wert:        & 1           & Information:     & Typ 11, Code 0 \\ 
			\hline \hline
			\multicolumn{2}{|c|}{\textbf{Letzte Nachricht}} & \multicolumn{2}{c|}{\textbf{Antwort}} \\ \hline
			Sendeadresse:    & 192.168.1.1 & Sendeadresse:    & 192.168.5.254 \\ \hline
			Empfangsadresse: & 192.168.5.1 & Empfangsadresse: & 192.168.1.1 \\ \hline
			Nachrichtentyp:  & ICMP        & Nachrichtentyp:  & ICMP \\ \hline
			TTL-Wert:        & 5           & Information:     & Typ 11, Code 0 \\ \hline
			 
		\end{tabular}
	\end{itemize}
\end{enumerate}

\pointsttl{3}


\end{document}
