\documentclass{article}
\usepackage{microtype}
\usepackage[utf8]{inputenc} 
\usepackage[a4paper, total={6in, 9.6in}]{geometry}
\usepackage{MnSymbol}
\usepackage{enumerate}
\usepackage{amsmath}
\usepackage{fancyhdr}
\usepackage{xcolor}
\usepackage{tikz}
\usepackage{pgfplots}
\usepackage{marvosym}

\widowpenalties=4 10000 10000 150 0

%% headers
\pagestyle{fancy}
\fancyhf{}
\rhead{Kommunikationssysteme WS19/20}
\lhead{Daniel Schubert, Anton Lydike}
\rfoot{Seite \thepage}

% simple command to display Aufgabe <num>)       ___ / <num>p.
\newcommand\task[1]{\section*{Aufgabe #1)\hfill \underline{\,\,\,\,\,\,}\,\,/1p.}}

% Interpretation (I)
\newcommand\I{I}
% Interpretation und belegung (I, \beta)
\newcommand\Ib{\I, \beta}

%% models
\newcommand\lmodels{\leftmodels} 			% =|
\newcommand\bimodels{\leftmodels\models}	% =||=


%% table for total points
\newcommand\pointsttl[1]{\section*{Gesamtpunkte: \hfill \underline{\,\,\,\,\,\,}\,\,/#1p.}}

%% Funktionen und Prädikate
% Funktionen (arg ist anzahl der stellen)
\newcommand\func[1]{\mathcal{F}^{#1}}
% Prädikate (arg ist anzahl der stellen)
\newcommand\praed[1]{\mathcal{P}^{#1}}

%% Regeln
\newcommand\defrule[2]{\frac{#1}{#2}}

%% Funktionszahl
\newcommand\funcnum[1]{\#_{F}\, #1}

% Für ersetzungen in belegungen wie { x \mapsto d }
\newcommand\repl[2]{\{#1 \mapsto #2\}}

% für alle x .
\newcommand\fall[1]{\forall #1 \, . \,}
\newcommand\ex[1]{\exists #1 \, . \,}

% short biimplication
\newcommand\biimpl{\Leftrightarrow}

% draw a box on the right side of the page
\newcommand\qed{ \hfill $\Box$ }

% red, green, blue text:

\definecolor{greeen}{RGB}{34,139,34}

\newcommand\red[1]{\textcolor{red}{#1}}
\newcommand\green[1]{\textcolor{greeen}{#1}}
\newcommand\blue[1]{\textcolor{blue}{#1}}

% more symbols: https://oeis.org/wiki/List_of_LaTeX_mathematical_symbols

\newcommand\cfgtitle[1]{\title{\vspace{-1.5cm}Übungsblatt #1\\%
\begin{large} Übungsgruppe Metcalfe \end{large}} \lfoot{Übungsblatt #1}\cfoot{Übungsgruppe Metcalfe}}
\author{Daniel Schubert\\Anton Lydike}


%%%%%%%%%%%%%%%%%%%%%%%
%% plotting helpers  %%
%%%%%%%%%%%%%%%%%%%%%%%

%% these draw vertical features
\newcommand\htl[1]{(#1,1) (#1,-1)}  		%% draw line from low to high
\newcommand\lth[1]{(#1,-1) (#1,1)}			%% draw line from high to low

\newcommand\sigtick[2]{\htl{#1} \lth{#2}}	%% draw a htl and then lth line

%% these draw horizontal features
\newcommand\sig[3]{(#2,#1) (#3,#1)}		%% draw a line at height #1 from x=#2 to x=#3
\newcommand\sighi[2]{\sig{1}{#1}{#2}}		%% draw a high signal from #1 to #2
\newcommand\sigmed[2]{\sig{0}{#1}{#2}}		%% draw a null signal from #1 to #2
\newcommand\siglo[2]{\sig{-1}{#1}{#2}}		%% draw a low  signal from #1 to #2


\newcommand\fakeaxis[2]{\addplot [-stealth,black] coordinates {(#1,0) (#2,0)};}



%% units
\newcommand\m{\text{ m}}
\newcommand\s{\text{ s}}
\newcommand\mps{\frac{\text{m}}{\text{s}}}
\newcommand\Gbps{\text{ Gbps}}
\newcommand\bps{\text{ bps}}
\newcommand\bit{\text{ b}}
\newcommand\B{\text{ B}}


\pgfplotsset{compat=1.15}

\renewcommand{\arraystretch}{1.5}

\cfgtitle{8}
\date{Mittwoch 18.12.2019}

\begin{document}
\maketitle
\thispagestyle{fancy}

\task{1}
\begin{enumerate}[a)]
	\item \begin{itemize}
		\item  \textbf{TCP-Flusskontrolle:} Auf Netzebene ist der TCP-Empfänger vor einem zu großen Zufluss von Segmenten eines TCP-Senders zu schützen
		\item \textbf{TCP-Überlastkontrolle:} Ende-zu-Ende-Mechanismus, um Stausituationen zu vermeiden und Auswirkungen von Staus zu begrenzen
	\end{itemize}
	\item Empfänger teilt Sender über das Header-Feld \textit{window-size} die aktuelle größe des Empfangsfensters mit.
	\item \textit{Window-Size = 0} blockiert vorläufig den sendenden Prozess und führt zu der Anpassung der \textit{Maximum Segment Size} (MSS). Dieser zustand wird durch die Mithilfe der Zero-Window-Probe behoben.
	\item Das Congestion Window wird beim Sender mitgeführt und beschränkt die Rate, mit der ein Sender Verkehr ins Netz senden kann.
	\item \begin{itemize}
		\item ACKs kommen an, \textit{window-size} bleibt gleich $\Rightarrow$ Send\_Window nach rechts.
		\item ACKs kommen an, \textit{window-size} ist 0 $\Rightarrow$ Senden wird blockiert.
		\item Drei gleiche ACKs kommen an $\Rightarrow$ Fast Recovery.
	\end{itemize}
\end{enumerate}

\task{2}
\begin{enumerate}[a)]
	\item 
	\item \hfill \begin{center}
		\includegraphics{{2b.plot}.pdf}
	\end{center}
	\item \hfill \begin{center}
		\includegraphics{{2c.plot}.pdf}
	\end{center}
	Nach dem $2\cdot \text{RTT}$ vergangen ist, zeigt \textit{LastByteSent} auf Byte 7999 und \textit{LastByteAcked} auf Byte 3999.
\end{enumerate}

\task{3}
\begin{enumerate}[a)]
	\item \begin{itemize}
		      \item 8.8.8.8 ist \textbf{keine} Adresse die für die Nutzung in privaten Netzwerken freigegeben ist.
		      \item \textbf{Ja}, da 127.0.126.1 eine sog. loopback-Adresse ist.
		      \item 192.168.1.255/24 ist \textbf{keine} Broadcast-Adresse im \texttt{/19}-er Raum, da das dritte oktett Nullen enthält, wo laut netzmaske nur einsen sein sollten.
		      \item 137.250.172.0/19 ist \textbf{keine} Netzadresse, da das dritte oktett (\texttt{10101100}) noch Einsen enthält, wo laut Netzmaske schon keine mehr sein sollten.
	      \end{itemize}
	\item Eine CIDR-Rotation von 123.122.121.120/19 erzeugt:
	      \begin{itemize}
		      \item Die Subnetzmaske 255.255.224.0
		      \item Die Netzadresse 123.122.96.0, die man erhält, wenn man die host ID mit der Subnetzmaske und-verknüpft (123.122.121.120 \& 255.255.224.0)
		      \item Und die Broadcast-Adresse 123.122.127.255
	      \end{itemize}
	\item \begin{itemize}
		      \item \texttt{::f} ist \textbf{eine} valide IPv6 Adresse.
		      \item \texttt{ff01::fb} ist \textbf{eine} valide IPv6 Adresse.
		      \item \texttt{fe:80:02:02:b3:13} ist \textbf{keine} valide IPv6 Adresse, da zwei Segmente a 16 bit fehlen.
		      \item \texttt{2019:4g:2020:e1f2::13} ist \textbf{keine} valide IPv6 Adresse, da \texttt{g} keine Hexadezimalzahl ist.
	      \end{itemize}
	\item Die gegebene IPv6-Adresse \texttt{2001:0db8:0000:0067:0230:0000:0001:ff00} ist
	    \begin{itemize}
			\item in kurzform geschrieben \texttt{2001:db8:0:67:230:0:1:ff00}
			\item eine Global-Unicast-Adresse
	    \end{itemize}
	\item Link-Local-Adressen sind nur innerhalb des lokalen Netzwerks gültig
\end{enumerate}



% 11111111.11111111.11100000.00000000
%!00000000.00000000.00011111.11111111
% 123     .122     .01111001.**
% 123     .122     .01111111.11111111

\pointsttl{3}


\end{document}
