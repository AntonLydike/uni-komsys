\documentclass{article}
\usepackage{microtype}
\usepackage[utf8]{inputenc} 
\usepackage[a4paper, total={6in, 9.6in}]{geometry}
\usepackage{MnSymbol}
\usepackage{enumerate}
\usepackage{amsmath}
\usepackage{fancyhdr}
\usepackage{xcolor}
\usepackage{tikz}
\usepackage{pgfplots}
\usepackage{marvosym}

\widowpenalties=4 10000 10000 150 0

%% headers
\pagestyle{fancy}
\fancyhf{}
\rhead{Kommunikationssysteme WS19/20}
\lhead{Daniel Schubert, Anton Lydike}
\rfoot{Seite \thepage}

% simple command to display Aufgabe <num>)       ___ / <num>p.
\newcommand\task[1]{\section*{Aufgabe #1)\hfill \underline{\,\,\,\,\,\,}\,\,/1p.}}

% Interpretation (I)
\newcommand\I{I}
% Interpretation und belegung (I, \beta)
\newcommand\Ib{\I, \beta}

%% models
\newcommand\lmodels{\leftmodels} 			% =|
\newcommand\bimodels{\leftmodels\models}	% =||=


%% table for total points
\newcommand\pointsttl[1]{\section*{Gesamtpunkte: \hfill \underline{\,\,\,\,\,\,}\,\,/#1p.}}

%% Funktionen und Prädikate
% Funktionen (arg ist anzahl der stellen)
\newcommand\func[1]{\mathcal{F}^{#1}}
% Prädikate (arg ist anzahl der stellen)
\newcommand\praed[1]{\mathcal{P}^{#1}}

%% Regeln
\newcommand\defrule[2]{\frac{#1}{#2}}

%% Funktionszahl
\newcommand\funcnum[1]{\#_{F}\, #1}

% Für ersetzungen in belegungen wie { x \mapsto d }
\newcommand\repl[2]{\{#1 \mapsto #2\}}

% für alle x .
\newcommand\fall[1]{\forall #1 \, . \,}
\newcommand\ex[1]{\exists #1 \, . \,}

% short biimplication
\newcommand\biimpl{\Leftrightarrow}

% draw a box on the right side of the page
\newcommand\qed{ \hfill $\Box$ }

% red, green, blue text:

\definecolor{greeen}{RGB}{34,139,34}

\newcommand\red[1]{\textcolor{red}{#1}}
\newcommand\green[1]{\textcolor{greeen}{#1}}
\newcommand\blue[1]{\textcolor{blue}{#1}}

% more symbols: https://oeis.org/wiki/List_of_LaTeX_mathematical_symbols

\newcommand\cfgtitle[1]{\title{\vspace{-1.5cm}Übungsblatt #1\\%
\begin{large} Übungsgruppe Metcalfe \end{large}} \lfoot{Übungsblatt #1}\cfoot{Übungsgruppe Metcalfe}}
\author{Daniel Schubert\\Anton Lydike}


%%%%%%%%%%%%%%%%%%%%%%%
%% plotting helpers  %%
%%%%%%%%%%%%%%%%%%%%%%%

%% these draw vertical features
\newcommand\htl[1]{(#1,1) (#1,-1)}  		%% draw line from low to high
\newcommand\lth[1]{(#1,-1) (#1,1)}			%% draw line from high to low

\newcommand\sigtick[2]{\htl{#1} \lth{#2}}	%% draw a htl and then lth line

%% these draw horizontal features
\newcommand\sig[3]{(#2,#1) (#3,#1)}		%% draw a line at height #1 from x=#2 to x=#3
\newcommand\sighi[2]{\sig{1}{#1}{#2}}		%% draw a high signal from #1 to #2
\newcommand\sigmed[2]{\sig{0}{#1}{#2}}		%% draw a null signal from #1 to #2
\newcommand\siglo[2]{\sig{-1}{#1}{#2}}		%% draw a low  signal from #1 to #2


\newcommand\fakeaxis[2]{\addplot [-stealth,black] coordinates {(#1,0) (#2,0)};}



%% units
\newcommand\m{\text{ m}}
\newcommand\s{\text{ s}}
\newcommand\mps{\frac{\text{m}}{\text{s}}}
\newcommand\Gbps{\text{ Gbps}}
\newcommand\bps{\text{ bps}}
\newcommand\bit{\text{ b}}
\newcommand\B{\text{ B}}


\pgfplotsset{compat=1.15}

\renewcommand{\arraystretch}{1}

\newcommand\tunderset[2]{\underset{\text{#1}}{\text{#2}}}

\cfgtitle{10}
\date{Mittwoch 22.1.2020}

\begin{document}
\maketitle
\thispagestyle{fancy}

\task{1}
\begin{enumerate}[a)]
	\item \begin{itemize}
		\item Ein Paket adressiert an 172.20.33.66 wird vom Router R1 über das Interface \texttt{if.B} an das Zielsubnetz weitergeleitet.
		\item Ein Paket adressiert an 172.20.72.36 wird vom Router R1 über das Interface \texttt{if.R3} an den Standard Gateway 172.20.1.10 gesendet.
		\item Ein Paket adressiert an 172.20.67.67 wird vom Router R1 über das Interface \texttt{if.R2} an den Router R2 mit IP 172.20.1.6 weitergeleitet.
	\end{itemize}
	\item Die Routingtabelle von Router R1: 
	\begin{center}\begin{tabular}{l|l|l|r}
		\textbf{Destination}  & \textbf{Genmask}       & \textbf{Gateway}     & \textbf{Iface}  \\ \hline
		172.20.32.0  & 255.255.255.0 & 0.0.0.0     & if.A   \\
		172.20.33.0  & 255.255.255.0 & 0.0.0.0     & if.B   \\
		172.20.34.0  & 255.255.255.0 & 0.0.0.0     & if.C   \\
		172.20.35.0  & 255.255.255.0 & 0.0.0.0     & if.D   \\
		172.20.128.0 & 255.255.255.0 & 172.20.1.18 & if.R4  \\
		172.20.64.0  & 255.255.252.0 & 172.20.1.6  & if.R2  \\
		default      & 0.0.0.0       & 172.20.1.10 & if.R3
	\end{tabular}\end{center}

	Kann kompakt gepublished werden:

	\begin{center}\begin{tabular}{l|l|l}
		Destination  & Genmask       & Gateway     \\ \hline
		172.20.32.0  & 255.255.252.0 & 0.0.0.0     \\
		172.20.128.0 & 255.255.255.0 & 172.20.1.18 \\
		172.20.64.0  & 255.255.252.0 & 172.20.1.6  
	\end{tabular}\end{center}
	\item Der \emph{Routing-Algorithmus}, der Open Shortest Path First (OSPF) zugrunde liegt, 
	ist Dijkstra, während das \emph{Routing-Protokoll} den gesamten Prozess beschreibt, auch 
	z.B. die Link-State-Advertisments. \textbf{!reformulate}
	\item Zwei vorteile von OSPF gegenüber RIP sind: \begin{itemize}
		\item RIP ist begrenzt auf eine entfernung von 15 hops.
		\item RIP reagiert nicht schnell auf Änderungen im Netzwerk.
	\end{itemize}
\end{enumerate}

\task{2}
\begin{enumerate}[a)]
	\item Ein \emph{AS} ist ein zusammenschluss von Netzwerken über Router, die unter einer Administration stehen. Änderungen innerhlab eines AS sind normalerweise nicht relevant außerhalb selbigens.
	\item Der Hauptzweck des  \emph{Border Gateway Protokolls (BGP)} ist die verbindung der einzelnen AS. Somit bildet das BGP den Kern des Internets.
	\item Das BGP ist \emph{Policy-Basiert}, da für jede Verbindung eine Policy für den Austausch der Routen aufgestellt werden muss.
	\item \begin{itemize}
		\item Router $C2$ erfährt über eBGP durch Router $F2$ von Präfix $z$, da $F2$ seine eigenen routen zu seinem Provider exportiert.
		\item Router $C2$ erfährt über OSPF durch Router $C3$ von Präfix $w$, da AS D 
		\item AS B kündigt AS C die BGP-Route [$y$; B–E]
		\item Von Subnetz $y$ zu Subnetz $z$ nehmen Pakete die AS-Route E-B-C-F
		\item Von Subnetz $y$ zu Subnetz $w$ nehmen Pakete die AS-Route E-B-C-D 
	\end{itemize}
\end{enumerate}

\task{3}
\begin{enumerate}[a)]
	\item Zwei Gründe für die Entwicklung von MLPS sind:
	\begin{itemize}
		\item Beschleunigte Vermittlung und Weiterleitung von Dateneinheiten.
		\item Differenzierte Behandlung von Datenströmen und vorhersagbare QoS.
	\end{itemize}
	\item { \hfill 
	\begin{tabular}{p{11.2cm}cc}
													& Korrekt & Falsch \\
		MPLS arbeitet zwischen den Schichten 
		2 und 3 des OSI-Schichtenmodells			& $\bigotimes$   & $\bigcirc$ \\ \\
		Der MPLS-Label enthält die IP-Zieladresse 
		der Label Edge Router						& $\bigcirc$   & $\bigotimes$     \\ \\
		An einem Datenpaket können gleichzeitig 
		mehrere Label angehängt werden				& $\bigotimes$   & $\bigcirc$     \\ \\
		Das Protokoll \emph{Forwarding Equivalent 
		Class} wird zur Signalisierung der MPLS
		Labels verwendet							& $\bigcirc$   & $\bigotimes$     
	\end{tabular}}
	\item Ein \emph{Label Edge Router} in einem MPLS-Netz ist die Schnitstelle zwischen MPLS-Netzen und \begin{itemize}
		\item Anderes AS-Netzen
		\item Subnetzen
		\item Anderen MPLS-Domänen im gleichen AS-Netz
	\end{itemize}
	\item -
\end{enumerate}

\pointsttl{3}
\end{document}