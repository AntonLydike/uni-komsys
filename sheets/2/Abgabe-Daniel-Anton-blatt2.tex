\documentclass{article}
\usepackage{microtype}
\usepackage[utf8]{inputenc} 
\usepackage[a4paper, total={6in, 9.6in}]{geometry}
\usepackage{MnSymbol}
\usepackage{enumerate}
\usepackage{amsmath}
\usepackage{fancyhdr}
\usepackage{xcolor}
\usepackage{tikz}
\usepackage{pgfplots}
\usepackage{marvosym}

\widowpenalties=4 10000 10000 150 0

%% headers
\pagestyle{fancy}
\fancyhf{}
\rhead{Kommunikationssysteme WS19/20}
\lhead{Daniel Schubert, Anton Lydike}
\rfoot{Seite \thepage}

% simple command to display Aufgabe <num>)       ___ / <num>p.
\newcommand\task[1]{\section*{Aufgabe #1)\hfill \underline{\,\,\,\,\,\,}\,\,/1p.}}

% Interpretation (I)
\newcommand\I{I}
% Interpretation und belegung (I, \beta)
\newcommand\Ib{\I, \beta}

%% models
\newcommand\lmodels{\leftmodels} 			% =|
\newcommand\bimodels{\leftmodels\models}	% =||=


%% table for total points
\newcommand\pointsttl[1]{\section*{Gesamtpunkte: \hfill \underline{\,\,\,\,\,\,}\,\,/#1p.}}

%% Funktionen und Prädikate
% Funktionen (arg ist anzahl der stellen)
\newcommand\func[1]{\mathcal{F}^{#1}}
% Prädikate (arg ist anzahl der stellen)
\newcommand\praed[1]{\mathcal{P}^{#1}}

%% Regeln
\newcommand\defrule[2]{\frac{#1}{#2}}

%% Funktionszahl
\newcommand\funcnum[1]{\#_{F}\, #1}

% Für ersetzungen in belegungen wie { x \mapsto d }
\newcommand\repl[2]{\{#1 \mapsto #2\}}

% für alle x .
\newcommand\fall[1]{\forall #1 \, . \,}
\newcommand\ex[1]{\exists #1 \, . \,}

% short biimplication
\newcommand\biimpl{\Leftrightarrow}

% draw a box on the right side of the page
\newcommand\qed{ \hfill $\Box$ }

% red, green, blue text:

\definecolor{greeen}{RGB}{34,139,34}

\newcommand\red[1]{\textcolor{red}{#1}}
\newcommand\green[1]{\textcolor{greeen}{#1}}
\newcommand\blue[1]{\textcolor{blue}{#1}}

% more symbols: https://oeis.org/wiki/List_of_LaTeX_mathematical_symbols

\newcommand\cfgtitle[1]{\title{\vspace{-1.5cm}Übungsblatt #1\\%
\begin{large} Übungsgruppe Metcalfe \end{large}} \lfoot{Übungsblatt #1}\cfoot{Übungsgruppe Metcalfe}}
\author{Daniel Schubert\\Anton Lydike}


%%%%%%%%%%%%%%%%%%%%%%%
%% plotting helpers  %%
%%%%%%%%%%%%%%%%%%%%%%%

%% these draw vertical features
\newcommand\htl[1]{(#1,1) (#1,-1)}  		%% draw line from low to high
\newcommand\lth[1]{(#1,-1) (#1,1)}			%% draw line from high to low

\newcommand\sigtick[2]{\htl{#1} \lth{#2}}	%% draw a htl and then lth line

%% these draw horizontal features
\newcommand\sig[3]{(#2,#1) (#3,#1)}		%% draw a line at height #1 from x=#2 to x=#3
\newcommand\sighi[2]{\sig{1}{#1}{#2}}		%% draw a high signal from #1 to #2
\newcommand\sigmed[2]{\sig{0}{#1}{#2}}		%% draw a null signal from #1 to #2
\newcommand\siglo[2]{\sig{-1}{#1}{#2}}		%% draw a low  signal from #1 to #2


\newcommand\fakeaxis[2]{\addplot [-stealth,black] coordinates {(#1,0) (#2,0)};}



%% units
\newcommand\m{\text{ m}}
\newcommand\s{\text{ s}}
\newcommand\mps{\frac{\text{m}}{\text{s}}}
\newcommand\Gbps{\text{ Gbps}}
\newcommand\bps{\text{ bps}}
\newcommand\bit{\text{ b}}
\newcommand\B{\text{ B}}


\newcommand\xor{\oplus}

\newcommand\kHz{\text{kHz}}
\newcommand\m{\text{m}}
\newcommand\s{\text{s}}

\DeclareMathOperator{\sinc}{sinc}

\cfgtitle{2}
\date{Donnerstag 07.11.2019}

\begin{document}
\maketitle
\thispagestyle{fancy}

\task{1}{1}

\begin{enumerate}[a)]
	\item Nein. 

	\item Nein.
	
	\item Nein.
	
\end{enumerate}

\task{2}{1}

\begin{enumerate}[a)]
	\item  
	
	\begin{itemize}
		\item Das Nyquist-Shannon-Abtasttheorem schreibt $f_A = 2 \times 14\kHz = 28\kHz$ vor.
		
		\item Es tritt der Alias-Effekt auf Damit können nur informationen bis $7\kHz$ vollständig rekonstruiert werden. Ein Signal mit einer frequenz von $14\kHz$ wird z.B. als konstanter wert gemessen und kann nicht vernünftig rekonstruiert werden.
		
		\item Es sind insgesamt 8 Bit zur Amplitudendiskretisierung verfügbar, dies ermöglicht theorethisch die darstellung von $2^8 = 256$ zuständen. Da es jedoch ein Vorzeichen-Bit gibt, haben wir die Werte $-0$ und $0$, welche identisch behandelt werden. Somit erhalten wir effektiv $255$ zustände.
	\end{itemize}

	\item
	
	\begin{itemize}
		\item Die Coderate des (7,4)-Hamming-Codes beträgt $\frac{4}{7} \approx 0.57$
			
		\item \texttt{0011 1111} $\underset{\text{(7,4)-H}}{\longmapsto}$ \texttt{0011\green{1}\blue{0}\red{1} 1111\green{0}\blue{0}\red{0}}
		
		\begin{itemize}
			\item $\green{p_1} = u_1 \xor u_2 \xor u_3$
			\item $\blue{p_2} = u_2 \xor u_3 \xor u_4$
			\item $\red{p_3} = u_1 \xor u_2 \xor u_4$
		\end{itemize}		
		
		\item Erkannt werden alle ein- und zwei-Bit Fehler. Korrigiert werden können nur ein-Bit Fehler.
	\end{itemize}
		
\end{enumerate}


\task{3}{1}

\begin{enumerate}[a)]

	\item 
	
	\begin{itemize}
	
		\item Nein
		\item Ja
		\item Nein, Frequenz wäre $\frac{1}{2\text{T}}$
	
	\end{itemize}

	\item Die \emph{Bandbreite} des Übertragungskanales ist definiert als der vorgegebene Frequenzbereich
	
	\item \emph{AWGN} wird modelliert mit $r(t) = s(t) + n(t)$. Es soll das Thermische Rauschen in elektronischen Bauteilen repräsentieren. Die einzelnen Terme sind folgendermaßen definiert:
	
	\begin{itemize}
	
		\item[$r(t)$] Das \textbf{empfangene} Signal
		\item[$s(t)$] Das \textbf{gesendete} Signal
		\item[$n(t)$] Sogenannte ,,Gaussian White Noise", also ein \textbf{gaußverteiltes, allfrequentes} Rauschen. Dieser Term wird einfach auf das gesendete Signal addiert, wie der Name suggeriert.
		
		
	\end{itemize}		
	
	
	\item \emph{Ausbreitungsverzögerung (Propagation Delay)} $t_{\text{p}}$ wird definiert als:
	
		$$ 
			t_{\text{p}} := \frac{d}{v \cdot c} = 
			\frac{\text{Leitungslänge in m}}{\text{Signalgeschwindigkeit in $\frac{\m}{\s}$}} 
		$$
		
		Damit ist die phsikalisch vorgeschrieben Verzögerung die zwischen dem Senden und dem Empfangen des Signales verstreicht gemeint, da die Signalgeschwindigkeit grundsätzlich auf Lichtgeschwindigkeit begrenzt ist.
	
	\item Das \emph{Nyquist-Shannon-Abtasttheorem} besagt, dass die Abtastfrequenz $f_A$ mindestens doppelt so hoch sein muss, wie die höchste im Signal vorkommende Frequenz $f_{max}$ um die verlustfreie rekonstruktion aus dem zeitdiskreten Signal zu garantieren ($f_A \geq 2f_{max}$).
	
	\bigskip
	
	Falls dies nicht gegeben ist, treten Artefakte auf (der sog. Alias-Effekt).
	
\end{enumerate}

\pointsttl{3}

\vfill

\end{document}
