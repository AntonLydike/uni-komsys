\documentclass{article}
\usepackage{microtype}
\usepackage[utf8]{inputenc} 
\usepackage[a4paper, total={6in, 9.6in}]{geometry}
\usepackage{MnSymbol}
\usepackage{enumerate}
\usepackage{amsmath}
\usepackage{fancyhdr}
\usepackage{xcolor}
\usepackage{tikz}
\usepackage{pgfplots}
\usepackage{marvosym}

\widowpenalties=4 10000 10000 150 0

%% headers
\pagestyle{fancy}
\fancyhf{}
\rhead{Kommunikationssysteme WS19/20}
\lhead{Daniel Schubert, Anton Lydike}
\rfoot{Seite \thepage}

% simple command to display Aufgabe <num>)       ___ / <num>p.
\newcommand\task[1]{\section*{Aufgabe #1)\hfill \underline{\,\,\,\,\,\,}\,\,/1p.}}

% Interpretation (I)
\newcommand\I{I}
% Interpretation und belegung (I, \beta)
\newcommand\Ib{\I, \beta}

%% models
\newcommand\lmodels{\leftmodels} 			% =|
\newcommand\bimodels{\leftmodels\models}	% =||=


%% table for total points
\newcommand\pointsttl[1]{\section*{Gesamtpunkte: \hfill \underline{\,\,\,\,\,\,}\,\,/#1p.}}

%% Funktionen und Prädikate
% Funktionen (arg ist anzahl der stellen)
\newcommand\func[1]{\mathcal{F}^{#1}}
% Prädikate (arg ist anzahl der stellen)
\newcommand\praed[1]{\mathcal{P}^{#1}}

%% Regeln
\newcommand\defrule[2]{\frac{#1}{#2}}

%% Funktionszahl
\newcommand\funcnum[1]{\#_{F}\, #1}

% Für ersetzungen in belegungen wie { x \mapsto d }
\newcommand\repl[2]{\{#1 \mapsto #2\}}

% für alle x .
\newcommand\fall[1]{\forall #1 \, . \,}
\newcommand\ex[1]{\exists #1 \, . \,}

% short biimplication
\newcommand\biimpl{\Leftrightarrow}

% draw a box on the right side of the page
\newcommand\qed{ \hfill $\Box$ }

% red, green, blue text:

\definecolor{greeen}{RGB}{34,139,34}

\newcommand\red[1]{\textcolor{red}{#1}}
\newcommand\green[1]{\textcolor{greeen}{#1}}
\newcommand\blue[1]{\textcolor{blue}{#1}}

% more symbols: https://oeis.org/wiki/List_of_LaTeX_mathematical_symbols

\newcommand\cfgtitle[1]{\title{\vspace{-1.5cm}Übungsblatt #1\\%
\begin{large} Übungsgruppe Metcalfe \end{large}} \lfoot{Übungsblatt #1}\cfoot{Übungsgruppe Metcalfe}}
\author{Daniel Schubert\\Anton Lydike}


%%%%%%%%%%%%%%%%%%%%%%%
%% plotting helpers  %%
%%%%%%%%%%%%%%%%%%%%%%%

%% these draw vertical features
\newcommand\htl[1]{(#1,1) (#1,-1)}  		%% draw line from low to high
\newcommand\lth[1]{(#1,-1) (#1,1)}			%% draw line from high to low

\newcommand\sigtick[2]{\htl{#1} \lth{#2}}	%% draw a htl and then lth line

%% these draw horizontal features
\newcommand\sig[3]{(#2,#1) (#3,#1)}		%% draw a line at height #1 from x=#2 to x=#3
\newcommand\sighi[2]{\sig{1}{#1}{#2}}		%% draw a high signal from #1 to #2
\newcommand\sigmed[2]{\sig{0}{#1}{#2}}		%% draw a null signal from #1 to #2
\newcommand\siglo[2]{\sig{-1}{#1}{#2}}		%% draw a low  signal from #1 to #2


\newcommand\fakeaxis[2]{\addplot [-stealth,black] coordinates {(#1,0) (#2,0)};}



%% units
\newcommand\m{\text{ m}}
\newcommand\s{\text{ s}}
\newcommand\mps{\frac{\text{m}}{\text{s}}}
\newcommand\Gbps{\text{ Gbps}}
\newcommand\bps{\text{ bps}}
\newcommand\bit{\text{ b}}
\newcommand\B{\text{ B}}


\usepackage{subcaption} 

\newcommand\kHz{\text{ kHz}}
\newcommand\m{\text{ m}}
\newcommand\s{\text{ s}}
\newcommand\bit{\text{ b}}
\newcommand\T{\text{T}}
\newcommand\F{\text{F}}
\newcommand\C{\text{C}}
\newcommand\B{\text{B}}
\newcommand\N{\text{N}}
\newcommand\W{\text{ W}}
\newcommand\dB{\text{ dB}}
\newcommand\dBm{\text{ dBm}}

\cfgtitle{3}
\date{Mittwoch 13.11.2019}

\newcommand\degr{^\circ}

\newcommand\ptsqr[2]{(#1, #2) (#1+1, #2) (#1, #2+1) (#1+1, #2+1)}

\begin{document}
\maketitle
\thispagestyle{fancy}

\task{1}

\begin{enumerate}[a)]
	\item Zu übertragende Bitfolge: \texttt{1001 1101}
	
	
\begin{itemize}
	\item \textit{NRZ-Code:} \\
		\begin{tikzpicture}
	        \begin{axis}[
	          domain=0:8.2,
	          xmin=0, xmax=8.2,
	          axis y line*=left,      
	          axis x line*=top,
		      x axis line style={draw opacity=0},
	          height=3cm, width=.8\textwidth,
	          yticklabels={Low, High},
	          ytick={-1, 1},
	          xticklabels={1,,0,,0,,1,,1,,1,,0,,1},
	          xtick={0.5,1,1.5,2,...,8},
	          xtick style={draw=none}
	          ]
	          
			\fakeaxis{0}{8.2}
	          
			\addplot [blue] coordinates {
				\sighi{0}{1}
				\siglo{1}{3}
				\sighi{3}{6}
				\siglo{6}{7}
				\sighi{7}{8}
			};
		\end{axis}
    	\end{tikzpicture}
	
	\item \textit{Manchester-Code:} 
	
		\begin{tikzpicture}
	        \begin{axis}[
	          domain=0:8.2,
	          xmin=0, xmax=8.2,
	          axis y line*=left,      
	          axis x line*=top,
		      x axis line style={draw opacity=0},
	          height=3cm, width=.8\textwidth,
	          yticklabels={Low, High},
	          ytick={-1, 1},
	          xticklabels={1,,0,,0,,1,,1,,1,,0,,1},
	          xtick={0.5,1,1.5,2,...,8},
	          xtick style={draw=none}
	          ]
	          
			\fakeaxis{0}{8.2}
	          
			\addplot [blue] coordinates {
				(0,1)
				\htl{0.5}
				\lth{1.5}
				\htl{2}
				\lth{2.5}
				\htl{3.5}
				\lth{4}
				\htl{4.5}
				\lth{5}
				\htl{5.5}
				\lth{6.5}
				\htl{7.5}
				(8, -1)
			};
		\end{axis}
    	\end{tikzpicture}
	
	\item \textit{Differentieller Manchester-Code:}	
	
		\begin{tikzpicture}
	        \begin{axis}[
	          domain=0:8.2,
	          xmin=0, xmax=8.2,
	          axis y line*=left,      
	          axis x line*=top,
		      x axis line style={draw opacity=0},
	          height=3cm, width=.8\textwidth,
	          yticklabels={Low, High},
	          ytick={-1, 1},
	          xticklabels={1,,0,,0,,1,,1,,1,,0,,1},
	          xtick={0.5,1,1.5,2,...,8},
	          xtick style={draw=none}
	          ]
	          
			\fakeaxis{0}{8.2}
	        
			\addplot [blue] coordinates {
				(0,1) 
				\sigtick{0.5}{1}
				\sigtick{1.5}{2}
				\sigtick{2.5}{3.5}
				\sigtick{4.5}{5.5}
				\sigtick{6}{6.5}
				\htl{7.5}
				(8, -1)
			};
		\end{axis}
    	\end{tikzpicture}	
	
		\item \textit{MLT-3-Code:}
	
			\begin{tikzpicture}
	        \begin{axis}[
	          domain=0:8.2,
	          xmin=0, xmax=8.2,
	          axis y line*=left,      
	          axis x line*=top,
		      x axis line style={draw opacity=0},
	          height=3cm, width=.8\textwidth,
	          yticklabels={Low,Null,High},
	          ytick={-1,0, 1},
	          xticklabels={1,,0,,0,,1,,1,,1,,0,,1},
	          xtick={0.5,1,1.5,2,...,8},
	          xtick style={draw=none}
	          ]
	          
			\fakeaxis{0}{8.2}
	    	      
				\addplot [blue] coordinates {
					\sighi{0}{3}
					\sigmed{3}{4}
					\siglo{4}{5}
					\sigmed{5}{7}
					\sighi{7}{8}
				};
			\end{axis}	
		\end{tikzpicture}
	
	\end{itemize}
	
	\item
	
	\begin{itemize}
		\item Nein.
		\item Ja.
		\item Ja.
		\item Nen, da die ,,Art der verwendeten Grundimpulse und deren Abfolge`` keine Einfluss auf den verwendeten Frequenzbereich haben sollte.
	\end{itemize}
\end{enumerate}

\task{2}


\begin{enumerate}[a)]
	\item 
		\begin{itemize}
			\item Hier handelt es sich um FSK
			\item FSK ist daran zu erkennen, dass die Frequenz des Trägersignales verändert wird um verschiedene zustände im Datensignal dar zu stellen.
		\end{itemize}

	\item

	\pagebreak

	\begin{itemize}
		\item Das Konstellationsdiagramm sieht aus wie folgt:
			
		\begin{figure}[!ht] 
		\centering
		\begin{tikzpicture}
			\pgfplotsset{set layers}
	        \begin{axis}[
	          xmin=-3, xmax=3,
	          ymin=-3, ymax=3,
	          axis y line*=center,
	          axis x line*=center,
	          height=4cm, width=4cm,
	          yticklabels={},
	          xticklabels={},
	          tick style={draw=none},
	          xlabel=180,
			  ylabel=270,
    	      every axis y label/.style={at=(current axis.below origin),anchor=north},
	          every axis x label/.style={at=(current axis.left of origin),anchor=east}  
	          ]
	          
	          \addplot[blue, mark=*, only marks] coordinates {
				\ptsqr{0.5}{0.5}
				\ptsqr{-1.5}{0.5}
				\ptsqr{-1.5}{-1.5}
				\ptsqr{0.5}{-1.5}
	          };
	        
			\end{axis}
		
	        \begin{axis}[
	          xmin=-3, xmax=3,
	          ymin=-3, ymax=3,
	          axis y line*=center,
	          axis x line*=center,
	          axis line style={draw=none},
	          height=4cm, width=4cm,
	          yticklabels={},
	          xticklabels={},
	          tick style={draw=none},
	          xlabel=0,
			  ylabel=90,
    	      every axis y label/.style={at=(current axis.above origin),anchor=south},
	          every axis x label/.style={at=(current axis.right of origin),anchor=west}
	          ]
	          \end{axis}
    	\end{tikzpicture}	
    	\end{figure}
		
		\item 25\% und 75\%
			
		\item Folgende Winkel definieren folgende Zustände:
		\begin{figure}[!ht] 
			\centering
			\begin{subfigure}[b]{0.4\textwidth}
		\begin{align*}22,5\degr &\to 0001 		\\
			45° &\to 0000, 0011		\\
			67,5\degr &\to 0010		\\
			112,5\degr &\to 1001		\\
			135\degr &\to 1000, 1011		\\
			157,5\degr &\to 1010		
		\end{align*}
			\end{subfigure}
			\begin{subfigure}[b]{0.4\textwidth}
		\begin{align*}202,5\degr &\to 1101		\\
			225\degr &\to 1100, 1111		\\
			247,5\degr &\to 1110		\\
			292,5\degr &\to 0101		\\
			315\degr &\to 0100, 0111		\\
			337,5\degr &\to 0110
		\end{align*}
			\end{subfigure}
		\end{figure}
	\end{itemize}
	
	\item Gegeben sei $\T = 100 \mu \s$ im ein 64-QAM-Verfahren, also ist $\N = 64\bit$
	
		\begin{align*}
			\frac{1}{\T} \cdot \log_2(\N)	&= \frac{1}{100\mu \s} \cdot \log_2(64\bit) \\
											&= \frac{1}{10^{-4} \s} \cdot 6\bit \\
											&= 6 \cdot 10^{4}\text{ bps} \\
											&= 60 \text{ kbps} \\
		\end{align*}

	\item Angenommen eine Bitrate von 100 Mbps und ein 16-QAM-Verfahren. Daraus folgt, dass $\T = 25 \mu \s$. Das ergibt eine Baudrate von $$\frac{1}{\T} = 25\cdot 10^6 \text{ Baut} = 25 \text{ MBaut}$$

\end{enumerate}

\task{3}

\begin{enumerate}[a)]
	\item 
		\begin{enumerate}
			\item[] Rechner A)
				\begin{align*}
	\text{Ausbreitungsverzögerung } t_{p_A} &= \frac{d_A}{v_A} = \frac{3\m}{2\cdot 10^8 \frac{\m}{\s}} &= 1,5\cdot 10^{-8}\s \\	
	\text{Übertragungsverzögerung } t_{s_A} &= \frac{L}{R_A} = \frac{1.500 \cdot 8\bit}{16 \cdot 10^9 \frac{\bit}{\s}} &= 7,5\cdot 10^{-7}\s
				\end{align*}
				
			\item[] Rechner B)
				\begin{align*}
	\text{Ausbreitungsverzögerung } t_{p_B} &= \frac{d_B}{v_B} = \frac{10\m}{3\cdot 10^8 \frac{\m}{\s}} &= 3\cdot 10^{-7}\s \\	
	\text{Übertragungsverzögerung } t_{s_B} &= \frac{L}{R_B} = \frac{1.500 \cdot 8\bit}{600 \cdot 10^6 \frac{\bit}{\s}} &= 2\cdot 10^{-5}\s
				\end{align*}
		\end{enumerate}
		
		Insgesamt $2,1065\cdot 10^{-5}\s$
		
		\item 
			
			$$ \F = \frac{\text{P}}{\text{P}_e} \iff \text{P}_e = \frac{20\W}{120\dB} \approx 0,167 \W \approx 22,22\dBm $$	
		
		\item 
	
			\begin{itemize}
				\item $10 \dB$
				\item $ \text{Dämpfung [dB]} = 10 \cdot \log_{10}{\frac{1\W}{100\text{ mW}}} = 10\dB $
			\end{itemize}					
		
		\item  Die Bandbreite B des Übertragungsweges bestimmt die maximal mögliche Symbolrate:
		
			$$ \C = \B \cdot \log_2(1+\text{SNR}) \text{ bps} $$
			
			Die Stärke der Störungen $\text{SNR} = \frac{\text{P}_\text{signal}}{\text{P}_\text{noise}}$ begrenzt den maximalen Informationsgehalt eines Symbols
			
			
		\item 
		
			\begin{itemize}
				\item Richtig
				\item Richtig
				\item Falsch
			\end{itemize}
\end{enumerate}



\pointsttl{3}



\end{document}